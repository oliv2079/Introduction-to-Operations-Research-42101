%----------------------------------------------------------------------------------------
%	PACKAGES AND OTHER DOCUMENT CONFIGURATIONS
%----------------------------------------------------------------------------------------

\documentclass{article}

%%%%%%%%%%%%%%%%%%%%%%%%%%%%%%%%%%%%%%%%%
% Lachaise Assignment
% Structure Specification File
% Version 1.0 (26/6/2018)
%
% This template originates from:
% http://www.LaTeXTemplates.com
%
% Authors:
% Marion Lachaise & François Févotte
% Vel (vel@LaTeXTemplates.com)
%
% License:
% CC BY-NC-SA 3.0 (http://creativecommons.org/licenses/by-nc-sa/3.0/)
% 
%%%%%%%%%%%%%%%%%%%%%%%%%%%%%%%%%%%%%%%%%

%----------------------------------------------------------------------------------------
%	PACKAGES AND OTHER DOCUMENT CONFIGURATIONS
%----------------------------------------------------------------------------------------

\usepackage{amsmath,amsfonts,stmaryrd,amssymb} % Math packages

\usepackage{enumerate} % Custom item numbers for enumerations

\usepackage[ruled]{algorithm2e} % Algorithms

\usepackage[framemethod=tikz]{mdframed} % Allows defining custom boxed/framed environments

\usepackage{float}

\usepackage{listings} % File listings, with syntax highlighting
\lstset{
	basicstyle=\ttfamily, % Typeset listings in monospace font
}

%----------------------------------------------------------------------------------------
%	DOCUMENT MARGINS
%----------------------------------------------------------------------------------------

\usepackage{geometry} % Required for adjusting page dimensions and margins

\geometry{
	paper=a4paper, % Paper size, change to letterpaper for US letter size
	top=2.5cm, % Top margin
	bottom=3cm, % Bottom margin
	left=2.5cm, % Left margin
	right=2.5cm, % Right margin
	headheight=14pt, % Header height
	footskip=1.5cm, % Space from the bottom margin to the baseline of the footer
	headsep=1.2cm, % Space from the top margin to the baseline of the header
	%showframe, % Uncomment to show how the type block is set on the page
}

%----------------------------------------------------------------------------------------
%	FONTS
%----------------------------------------------------------------------------------------

\usepackage[utf8]{inputenc} % Required for inputting international characters
\usepackage[T1]{fontenc} % Output font encoding for international characters

\usepackage{XCharter} % Use the XCharter fonts

%----------------------------------------------------------------------------------------
%	COMMAND LINE ENVIRONMENT
%----------------------------------------------------------------------------------------

% Usage:
% \begin{commandline}
%	\begin{verbatim}
%		$ ls
%		
%		Applications	Desktop	...
%	\end{verbatim}
% \end{commandline}

\mdfdefinestyle{commandline}{
	leftmargin=10pt,
	rightmargin=10pt,
	innerleftmargin=15pt,
	middlelinecolor=black!50!white,
	middlelinewidth=2pt,
	frametitlerule=false,
	backgroundcolor=black!5!white,
	frametitle={Command Line},
	frametitlefont={\normalfont\sffamily\color{white}\hspace{-1em}},
	frametitlebackgroundcolor=black!50!white,
	nobreak,
}

% Define a custom environment for command-line snapshots
\newenvironment{commandline}{
	\medskip
	\begin{mdframed}[style=commandline]
}{
	\end{mdframed}
	\medskip
}

%----------------------------------------------------------------------------------------
%	FILE CONTENTS ENVIRONMENT
%----------------------------------------------------------------------------------------

% Usage:
% \begin{file}[optional filename, defaults to "File"]
%	File contents, for example, with a listings environment
% \end{file}

\mdfdefinestyle{file}{
	innertopmargin=1.6\baselineskip,
	innerbottommargin=0.8\baselineskip,
	topline=false, bottomline=false,
	leftline=false, rightline=false,
	leftmargin=2cm,
	rightmargin=2cm,
	singleextra={%
		\draw[fill=black!10!white](P)++(0,-1.2em)rectangle(P-|O);
		\node[anchor=north west]
		at(P-|O){\ttfamily\mdfilename};
		%
		\def\l{3em}
		\draw(O-|P)++(-\l,0)--++(\l,\l)--(P)--(P-|O)--(O)--cycle;
		\draw(O-|P)++(-\l,0)--++(0,\l)--++(\l,0);
	},
	nobreak,
}

% Define a custom environment for file contents
\newenvironment{file}[1][File]{ % Set the default filename to "File"
	\medskip
	\newcommand{\mdfilename}{#1}
	\begin{mdframed}[style=file]
}{
	\end{mdframed}
	\medskip
}

%----------------------------------------------------------------------------------------
%	NUMBERED QUESTIONS ENVIRONMENT
%----------------------------------------------------------------------------------------

% Usage:
% \begin{question}[optional title]
%	Question contents
% \end{question}

\mdfdefinestyle{question}{
	innertopmargin=1.2\baselineskip,
	innerbottommargin=0.8\baselineskip,
	roundcorner=5pt,
	nobreak,
	singleextra={%
		\draw(P-|O)node[xshift=1em,anchor=west,fill=white,draw,rounded corners=5pt]{%
		Question \theQuestion\questionTitle};
	},
}

\newcounter{Question} % Stores the current question number that gets iterated with each new question

% Define a custom environment for numbered questions
\newenvironment{question}[1][\unskip]{
	\bigskip
	\stepcounter{Question}
	\newcommand{\questionTitle}{~#1}
	\begin{mdframed}[style=question]
}{
	\end{mdframed}
	\medskip
}

%----------------------------------------------------------------------------------------
%	WARNING TEXT ENVIRONMENT
%----------------------------------------------------------------------------------------

% Usage:
% \begin{warn}[optional title, defaults to "Warning:"]
%	Contents
% \end{warn}

\mdfdefinestyle{warning}{
	topline=false, bottomline=false,
	leftline=false, rightline=false,
	nobreak,
	singleextra={%
		\draw(P-|O)++(-0.5em,0)node(tmp1){};
		\draw(P-|O)++(0.5em,0)node(tmp2){};
		\fill[black,rotate around={45:(P-|O)}](tmp1)rectangle(tmp2);
		\node at(P-|O){\color{white}\scriptsize\bf !};
		\draw[very thick](P-|O)++(0,-1em)--(O);%--(O-|P);
	}
}

% Define a custom environment for warning text
\newenvironment{warn}[1][Warning:]{ % Set the default warning to "Warning:"
	\medskip
	\begin{mdframed}[style=warning]
		\noindent{\textbf{#1}}
}{
	\end{mdframed}
}

%----------------------------------------------------------------------------------------
%	INFORMATION ENVIRONMENT
%----------------------------------------------------------------------------------------

% Usage:
% \begin{info}[optional title, defaults to "Info:"]
% 	contents
% 	\end{info}

\mdfdefinestyle{info}{%
	topline=false, bottomline=false,
	leftline=false, rightline=false,
	nobreak,
	singleextra={%
		\fill[black](P-|O)circle[radius=0.4em];
		\node at(P-|O){\color{white}\scriptsize\bf i};
		\draw[very thick](P-|O)++(0,-0.8em)--(O);%--(O-|P);
	}
}

% Define a custom environment for information
\newenvironment{info}[1][Info:]{ % Set the default title to "Info:"
	\medskip
	\begin{mdframed}[style=info]
		\noindent{\textbf{#1}}
}{
	\end{mdframed}
}
 % Include the file specifying the document structure and custom commands

%----------------------------------------------------------------------------------------
%	ASSIGNMENT INFORMATION
%----------------------------------------------------------------------------------------

\title{Homework 2: Integer Programming} % Title of the assignment

\author{Oliver Rutving s164209} % Author name and email address

\date{Introduction to Linear Programming  --- \today} % University, school and/or department name(s) and a date

%----------------------------------------------------------------------------------------


\begin{document}

\maketitle % Print the title

%----------------------------------------------------------------------------------------
%	PROBLEM 1
%----------------------------------------------------------------------------------------

\section{Question 1}
\subsection{Exercise 1}
Let $x_1, ..., x_7$ represent how many lots (in $m^3$) should be shipped to client 1-7 respectively. 
Lots cannot be divided. The integer linear program is formulated as so:

\begin{equation*}
  \begin{array}{ll@{}ll}
  \text{Maximize}  & Z=920*x_1+530*x_2+515*x_3+720*x_4+1127*x_5+730*x_6+1020*x_7\\
                   & \\
  \text{Subject to}& 10*x_1+8*x_2+6*x_3+9*x_4+15*x_5+10*x_6+12*x_7 \leq 1500\\
                   & x_1\leq 12 \\
                   & x_2\leq 31 \\
                   & x_3\leq 20 \\
                   & x_4\leq 25 \\
                   & x_5\leq 50 \\
                   & x_6\leq 40 \\
                   & x_7\leq 60 \\
                   & x_i\in \mathbb{Z_+}\ |\ i=1,...,7
  \end{array}
\end{equation*}


\subsection{Exercise 2}
The integer program is implemented in Julia which returns the following:

% Command-line "screenshot"
\begin{commandline}
\begin{verbatim}
Objective value: 124207.00000000001
x1 = 12.0
x2 = 0.0
x3 = 20.0
x4 = 25.0
x5 = 21.0
x6 = 0.0
x7 = 60.0
	\end{verbatim}
\end{commandline}


\subsection{Exercise 3}
To meet with the new regulations, a new constraint is added for each variable that forces all variables to be bigger than half of their respective requested number of lots.
The change is implemented in Julia and returns the following:

% Command-line "screenshot"
\begin{commandline}
  \begin{verbatim}
Objective value: 119280.0
x1 = 11.0
x2 = 16.0
x3 = 19.0
x4 = 13.0
x5 = 25.0
x6 = 20.0
x7 = 38.0
    \end{verbatim}
  \end{commandline}

It is observed that the objective value is now smaller compared to before the change.

\subsection{Exercise 4}

There is now a start up cost when loading lots at a client. 
This change requires a binary variable indicating whether a client is used or not.
Implementing the new requirement changes the linear integer model to the following:

\begin{equation*}
  \begin{array}{ll@{}ll}
  \text{Maximize}  & Z=920*x_1+530*x_2+515*x_3+720*x_4+1127*x_5+730*x_6+1020*x_7-300*y_1\\
                   & \ \ \ \ \ \ \ \ -160*y_2-240*y_3-400*y_4-600*y_5-250*y_6-280*y_7 \\
                   & \\
  \text{Subject to}& 10*x_1+8*x_2+6*x_3+9*x_4+15*x_5+10*x_6+12*x_7 \leq 1500\\
                   & x_1\leq 12*y_1 \\
                   & x_2\leq 31*y_2 \\
                   & x_3\leq 20*y_3 \\
                   & x_4\leq 25*y_4 \\
                   & x_5\leq 50*y_5 \\
                   & x_6\leq 40*y_6 \\
                   & x_7\leq 60*y_7 \\
                   & x_i\in \mathbb{Z_+}\ |\ i=1,...,7
                   & y_i\in {0,1}\ |\ i=1,...,7
  \end{array}
\end{equation*}


\begin{commandline}
  \begin{verbatim}
Objective value: 122387.0
x1 = 12.0
x2 = 0.0
x3 = 20.0
x4 = 25.0
x5 = 21.0
x6 = 0.0
x7 = 60.0
y = 1-dimensional DenseAxisArray{Float64,1,...} with index sets:
    Dimension 1, 0:6
And data, a 7-element Array{Float64,1}:      
  1.0
  0.0
  1.0
  1.0
  1.0
  0.0
  1.0
    \end{verbatim}
  \end{commandline}

The objective value has decreased. Client 2 and 6 are not used.

\section{Question 2}
\subsection{Exercise 1}

The problem is set up in Julia which returns the following:

\begin{commandline}
  \begin{verbatim}
    Objective Value: 7.0
    ...
    And data, a 15-element Array{Float64,1}:     
     1.0
     0.0
     0.0
     0.0
     1.0
     1.0
     0.0
     1.0
     0.0
     0.0
     1.0
     0.0
     1.0
     1.0
     0.0
    \end{verbatim}
  \end{commandline}

And so meeting request 1, 5, 6, 8, 11, 13, 14 is one combination of bookings to make sure the maximum number of meeting requests are fulfilled.
The maximum number of meeting requests is therefore 7.

\subsection{Exercise 2}
The change has been implemented in Julia, but it does not compute.

\subsection{Exercise 3}

\subsection{Exercise 4}

\subsection{Exercise 5}

\section{Question 3}

\subsection{Exercise 1}

Decision variables:

$x_{ij}=$ \{

1 if arc (i,j) is used in the solution

0 otherwise

\}

$u_i\geq 0, the position of node i in the solution$


\begin{equation*}
  \begin{array}{ll@{}ll}
  \text{Minimize}  & \sum_{i = 1}^{n}\sum_{j = 1}^{n} c_{ij}x_{ij}  &  \\
                   & \\
  \text{Subject to}& \sum_{p = 1}^{m} x_{1p}=1 & \\
                   & \sum_{b = 1}^{m*2} x_{pb}=1 & \forall p \\
                   & \sum_{b = 1}^{m*2}\sum{d = 1}^{m} x_{db}=m-1 & \\
                   & u_i-u_j+1<=(n-1)*(1-x_{ij}) & \forall i \neq 1, \forall j \neq 1\\
                   & u_1=1 & \\
                   & 2\leq u_i \leq m & \forall i=1,... , m
  \end{array}
\end{equation*}

Here, $d$ denotes all delivery points, $p$ denotes all pickup points and $b$ denotes both all delivery points and pickup points. 

In the first constraint we make sure that the office is connected to one of the pickup places.
In the second constraint we make sure that all pickup location is connected to either another pickup location or a delivery location.
The third constraint makes sure that all delivery points (except one being the end point) are connected with either a pickup point or another delivery point.
The fourth through sixth constraint makes sure that a path is made 

\subsection{Exercise 2}
\subsection{Exercise 3}
\subsection{Exercise 4}
\subsection{Exercise 5}


end of the assignment
\end{document}
